\section{Einleitung und Problembeschreibung}

%TODO z.B. Ziele der Implementierung: Compile-Time Flexbilität

Dieses Jahr bestand die Aufgabe des Praktikums darin einen Renderer zu schreiben der gegebene Szenen rendert und als Bitmap-Bild ausgibt.
Damit diese Aufgabe machbar bleibt wurde einige Einschränkungen getätigt.
In der ersten Phase des Praktikums ging es lediglich darum einen Witted-Style Raytracer mit nur diffusem Lichtanteil zu implementieren.
Auch die geometrischen Strukturen wurden auf nur Dreiecke begrenzt.

Anhand dieser Aufgabenstellung haben wir entschieden, dass wir eine möglichst flexible und erweiterbare Lösung schreiben wollen, da wir uns bewusst waren, dass im zweiten Teil der Aufgabe noch etwas dazukommen würde.
Eine bewusste Entscheidung war auch, dass wir für alle Features die wir implementieren die Möglichkeiten haben wollten sie zur Compilezeit an- und ausszuchalten. Diese Idee würde dann auch eine (fast) beliebige Kombination der Features zum testen zulassen.

Der Zweite Teil der Implementierungsphase war den Renderer um einen eingeschränkten Pathtracer zu ergänzen.
In dem ersten Teil der Implementierung hatten wir noch keinen CUDA-Tracer implementiert, da unsere CPU-Lösung schon sehr gute Ergebnisse erziehlte, jedoch wurde uns bewusst, dass dieser für den zweiten Teil von großer Wichtigkeit war.
Deshalb wurde im zweiten Teil der Pathtracer sowohl für die CPU als auch mit CUDA für die GPU geschrieben.
