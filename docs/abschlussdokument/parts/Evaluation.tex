\section{Auswertung}
Während der Implementierung haben wir immer beachtet, die Effekte jeder implementierten Beschleunigungstechnik überprüfen zu können.
Zur Compilezeit ist es uns daher u.a. möglich, die genutzte Beschleunigungsstruktur festzulegen und verschiedene andere Techniken an- bzw. abzuschalten.
\begin{figure}
  \center
  \includegraphics[width=.6\textwidth]{images/configuration.png}
  \caption{Konfigurationsübersicht unseres Raytracers}
  \label{fig:pray_config}
\end{figure}
In Abbildung \ref{fig:pray_config} illustriert eine mögliche Konfiguration, mit der unser Programm gebaut werden kann.
Diese Eigenschaft hat es uns ermöglicht, den genauen Einfluss auf die Ausführungsgeschwindigkeit nachmessen zu können.
Gemessen wurde dabei mit den in C++11 eigeführten Mitteln aus dem \code{std::chrono} Namespace.
Im folgenden möchten wir aufzeigen, wie die verschiedenen von uns implementierten Features die Leistung unseres Raytracers im einzelnen beeinflussen.

\subsection{Getestete Szenen}
\begin{figure}
  \center
  \includegraphics[width=.6\textwidth]{images/monkey.png}
  \caption{Monkey Szene}
  \label{fig:monkey}
\end{figure}
\begin{figure}
  \center
  \includegraphics[width=.6\textwidth]{images/plant.png}
  \caption{Plant Szene}
  \label{fig:plant}
\end{figure}
\begin{figure}
  \center
  \includegraphics[width=.6\textwidth]{images/pt_cornell.png}
  \caption{PT/Cornell Szene}
  \label{fig:pt_cornell}
\end{figure}
\ref{fig:monkey}
\ref{fig:plant}
\ref{fig:pt_cornell}
  
\subsection{Getestete Konfigurationen}
\subsection{Ergebnisse}
\begin{figure}
  \center
  \caption{Monkey Szene}
  \label{fig:runtime_monkey}
  \begin{tikzpicture}
  \begin{axis} [
    symbolic x coords={Baseline,Adaptive-Sampling,OpenMP,SSERay,BIH,KdTree,Kombiniert},
    table/header=false,
    enlargelimits=0.15,
    ymode=log,
    scaled y ticks = false,
    ylabel={Laufzeit (ms)},
    xtick=data,
    x tick label style={rotate=45,anchor=east},
  ]
      \addplot [box plot median] table {evaluation/monkey.dat};
      \addplot [box plot box] table {evaluation/monkey.dat};
      \addplot [box plot top whisker] table {evaluation/monkey.dat};
      \addplot [box plot bottom whisker] table {evaluation/monkey.dat};
  \end{axis}
  \end{tikzpicture}
\end{figure}

\begin{figure}
  \center
  \caption{Plant Szene}
  \label{fig:runtime_plant}
  \begin{tikzpicture}
  \begin{axis} [
    symbolic x coords={Baseline,Adaptive-Sampling,OpenMP,SSERay,BIH,KdTree,Kombiniert},
    table/header=false,
    enlargelimits=0.15,
    ymode=log,
    scaled y ticks = false,
    ylabel={Laufzeit (ms)},
    xtick=data,
    x tick label style={rotate=45,anchor=east},
  ]
      \addplot [box plot median] table {evaluation/plant.dat};
      \addplot [box plot box] table {evaluation/plant.dat};
      \addplot [box plot top whisker] table {evaluation/plant.dat};
      \addplot [box plot bottom whisker] table {evaluation/plant.dat};
  \end{axis}
  \end{tikzpicture}
\end{figure}

\begin{figure}
  \center
  \caption{PT/Cornell Szene}
  \label{fig:runtime_pt_cornell}
  \begin{tikzpicture}
  \begin{axis} [
    symbolic x coords={Baseline,Adaptive-Sampling,OpenMP,SSERay,Cuda,BIH,KdTree,Kombiniert},
    table/header=false,
    enlargelimits=0.15,
    ymode=log,
    scaled y ticks = false,
    ylabel={Laufzeit (ms)},
    xtick=data,
    x tick label style={rotate=45,anchor=east},
  ]
      \addplot [box plot median] table {evaluation/pt_cornell.dat};
      \addplot [box plot box] table {evaluation/pt_cornell.dat};
      \addplot [box plot top whisker] table {evaluation/pt_cornell.dat};
      \addplot [box plot bottom whisker] table {evaluation/pt_cornell.dat};
  \end{axis}
  \end{tikzpicture}
\end{figure}
