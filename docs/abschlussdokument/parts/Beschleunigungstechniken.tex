\section{Beschleunigungstechniken}
Um die bestmöglichste Leistung mit der vorhandenen Hardware zu erzielen ist es notwendig, deren spezielle Eigenschaften so weit es geht auszuschöpfen.
Somit wird beispielsweise durch das Vorhandensein mehrerer CPU-Kerne eine mehrfädige Ausführung unabdingbar.
Befehlssatzerweiterungen für das Rechnen mit Vektoren, z.B. SSE bzw. AVX können mit speziellen Funktionen, sogenannten Intrinsics, explizit verwendet werden.
Falls exterene Rechenbeschleuniger, wie z.B. Cuda-fähige Grafikkarten zur Verfügung stehen, können auch diese von der Implementierung miteinbezogen werden, um den Durchsatz des eigenen Programms zu steigern.
Alle genannten Aspekte haben wir in unserem Raytracer beachtet.
Im Folgenden soll daher näher auf Details bei der Umsetzung eingegangen werden.
